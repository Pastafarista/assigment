\chapter{L-Systems}

\section{¿Qué son los L-Systems?}

\noindent Un sistema-L o también denominados sistemas de Lindenmayer se trata de un conjunto de reglas y símbolos usados para modelar el proceso de crecimiento de las plantas.

\noindent Fueron introducidos en 1968 por Aristid Lindenmayer, el cual estudió los patrones de crecimiento de varias algas y tenían como objetivo realizar una descripción formal del desarrollo de organismos y representar la relación entre células de plantas.

\section{¿Cómo funcionan?}

\noindent Las reglas de estos sistemas se basa en la autosimilitud, por eso las figuras que se suelen modelar suelen representar formas de tipo fractal.

\noindent Los sistemas-L se definen como un conjunto $G=\{V,S, \omega ,P\}$

\begin{itemize}
    \item $V$ es un conjunto de símbolos que contiene elementos que pueden ser reemplazados.
    \item $S$ es un conjunto de símbolos que contiene elementos que se mantienen fijos.
    \item $\omega$ es una cadena de símbolos de $V$ que definen el estado inicial del sistema (inicio o axioma).
    \item $P$ es un conjunto de reglas o producciones que definen la forma en la que las variables pueden ser reemplazadas por combinaciones de constantes y otras variables.
\end{itemize}